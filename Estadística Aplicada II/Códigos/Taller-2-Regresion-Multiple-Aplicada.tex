% Options for packages loaded elsewhere
\PassOptionsToPackage{unicode}{hyperref}
\PassOptionsToPackage{hyphens}{url}
%
\documentclass[
]{article}
\usepackage{amsmath,amssymb}
\usepackage{lmodern}
\usepackage{iftex}
\ifPDFTeX
  \usepackage[T1]{fontenc}
  \usepackage[utf8]{inputenc}
  \usepackage{textcomp} % provide euro and other symbols
\else % if luatex or xetex
  \usepackage{unicode-math}
  \defaultfontfeatures{Scale=MatchLowercase}
  \defaultfontfeatures[\rmfamily]{Ligatures=TeX,Scale=1}
\fi
% Use upquote if available, for straight quotes in verbatim environments
\IfFileExists{upquote.sty}{\usepackage{upquote}}{}
\IfFileExists{microtype.sty}{% use microtype if available
  \usepackage[]{microtype}
  \UseMicrotypeSet[protrusion]{basicmath} % disable protrusion for tt fonts
}{}
\makeatletter
\@ifundefined{KOMAClassName}{% if non-KOMA class
  \IfFileExists{parskip.sty}{%
    \usepackage{parskip}
  }{% else
    \setlength{\parindent}{0pt}
    \setlength{\parskip}{6pt plus 2pt minus 1pt}}
}{% if KOMA class
  \KOMAoptions{parskip=half}}
\makeatother
\usepackage{xcolor}
\usepackage[margin=1in]{geometry}
\usepackage{color}
\usepackage{fancyvrb}
\newcommand{\VerbBar}{|}
\newcommand{\VERB}{\Verb[commandchars=\\\{\}]}
\DefineVerbatimEnvironment{Highlighting}{Verbatim}{commandchars=\\\{\}}
% Add ',fontsize=\small' for more characters per line
\usepackage{framed}
\definecolor{shadecolor}{RGB}{248,248,248}
\newenvironment{Shaded}{\begin{snugshade}}{\end{snugshade}}
\newcommand{\AlertTok}[1]{\textcolor[rgb]{0.94,0.16,0.16}{#1}}
\newcommand{\AnnotationTok}[1]{\textcolor[rgb]{0.56,0.35,0.01}{\textbf{\textit{#1}}}}
\newcommand{\AttributeTok}[1]{\textcolor[rgb]{0.77,0.63,0.00}{#1}}
\newcommand{\BaseNTok}[1]{\textcolor[rgb]{0.00,0.00,0.81}{#1}}
\newcommand{\BuiltInTok}[1]{#1}
\newcommand{\CharTok}[1]{\textcolor[rgb]{0.31,0.60,0.02}{#1}}
\newcommand{\CommentTok}[1]{\textcolor[rgb]{0.56,0.35,0.01}{\textit{#1}}}
\newcommand{\CommentVarTok}[1]{\textcolor[rgb]{0.56,0.35,0.01}{\textbf{\textit{#1}}}}
\newcommand{\ConstantTok}[1]{\textcolor[rgb]{0.00,0.00,0.00}{#1}}
\newcommand{\ControlFlowTok}[1]{\textcolor[rgb]{0.13,0.29,0.53}{\textbf{#1}}}
\newcommand{\DataTypeTok}[1]{\textcolor[rgb]{0.13,0.29,0.53}{#1}}
\newcommand{\DecValTok}[1]{\textcolor[rgb]{0.00,0.00,0.81}{#1}}
\newcommand{\DocumentationTok}[1]{\textcolor[rgb]{0.56,0.35,0.01}{\textbf{\textit{#1}}}}
\newcommand{\ErrorTok}[1]{\textcolor[rgb]{0.64,0.00,0.00}{\textbf{#1}}}
\newcommand{\ExtensionTok}[1]{#1}
\newcommand{\FloatTok}[1]{\textcolor[rgb]{0.00,0.00,0.81}{#1}}
\newcommand{\FunctionTok}[1]{\textcolor[rgb]{0.00,0.00,0.00}{#1}}
\newcommand{\ImportTok}[1]{#1}
\newcommand{\InformationTok}[1]{\textcolor[rgb]{0.56,0.35,0.01}{\textbf{\textit{#1}}}}
\newcommand{\KeywordTok}[1]{\textcolor[rgb]{0.13,0.29,0.53}{\textbf{#1}}}
\newcommand{\NormalTok}[1]{#1}
\newcommand{\OperatorTok}[1]{\textcolor[rgb]{0.81,0.36,0.00}{\textbf{#1}}}
\newcommand{\OtherTok}[1]{\textcolor[rgb]{0.56,0.35,0.01}{#1}}
\newcommand{\PreprocessorTok}[1]{\textcolor[rgb]{0.56,0.35,0.01}{\textit{#1}}}
\newcommand{\RegionMarkerTok}[1]{#1}
\newcommand{\SpecialCharTok}[1]{\textcolor[rgb]{0.00,0.00,0.00}{#1}}
\newcommand{\SpecialStringTok}[1]{\textcolor[rgb]{0.31,0.60,0.02}{#1}}
\newcommand{\StringTok}[1]{\textcolor[rgb]{0.31,0.60,0.02}{#1}}
\newcommand{\VariableTok}[1]{\textcolor[rgb]{0.00,0.00,0.00}{#1}}
\newcommand{\VerbatimStringTok}[1]{\textcolor[rgb]{0.31,0.60,0.02}{#1}}
\newcommand{\WarningTok}[1]{\textcolor[rgb]{0.56,0.35,0.01}{\textbf{\textit{#1}}}}
\usepackage{graphicx}
\makeatletter
\def\maxwidth{\ifdim\Gin@nat@width>\linewidth\linewidth\else\Gin@nat@width\fi}
\def\maxheight{\ifdim\Gin@nat@height>\textheight\textheight\else\Gin@nat@height\fi}
\makeatother
% Scale images if necessary, so that they will not overflow the page
% margins by default, and it is still possible to overwrite the defaults
% using explicit options in \includegraphics[width, height, ...]{}
\setkeys{Gin}{width=\maxwidth,height=\maxheight,keepaspectratio}
% Set default figure placement to htbp
\makeatletter
\def\fps@figure{htbp}
\makeatother
\setlength{\emergencystretch}{3em} % prevent overfull lines
\providecommand{\tightlist}{%
  \setlength{\itemsep}{0pt}\setlength{\parskip}{0pt}}
\setcounter{secnumdepth}{5}
\ifLuaTeX
  \usepackage{selnolig}  % disable illegal ligatures
\fi
\IfFileExists{bookmark.sty}{\usepackage{bookmark}}{\usepackage{hyperref}}
\IfFileExists{xurl.sty}{\usepackage{xurl}}{} % add URL line breaks if available
\urlstyle{same} % disable monospaced font for URLs
\hypersetup{
  pdftitle={Taller 2 Regresión lineal Multiple},
  pdfauthor={Andrés Felipe Palomino - David Stiven Rojas},
  hidelinks,
  pdfcreator={LaTeX via pandoc}}

\title{Taller 2 Regresión lineal Multiple}
\author{Andrés Felipe Palomino - David Stiven Rojas}
\date{2023-04-21}

\begin{document}
\maketitle

\hypertarget{introducciuxf3n}{%
\section{Introducción}\label{introducciuxf3n}}

La base de datos \("yarn"\) obtenida de la librería (PLS) contiene
información sobre espectros NIR y mediciones de densidad de hilos de
PET, consta de 28 individuos (hilos de PET), 268 variables predictoras
(NIRS) y una variable de respuesta (densidad). Se ajustará un modelo
lineal múltiple para estimar la densidad del hilo PET, mediante
mediciones NIR

\begin{Shaded}
\begin{Highlighting}[]
\CommentTok{\#Importación de librerías necesarias}
\FunctionTok{library}\NormalTok{(car)}
\FunctionTok{library}\NormalTok{(MASS)}
\FunctionTok{library}\NormalTok{(xtable)}
\FunctionTok{library}\NormalTok{(lmtest)}
\FunctionTok{library}\NormalTok{(readxl)}
\FunctionTok{library}\NormalTok{(lmridge)}
\FunctionTok{library}\NormalTok{(pls)}
\end{Highlighting}
\end{Shaded}

\hypertarget{base-de-datos}{%
\subsection{Base de datos}\label{base-de-datos}}

En la siguiente tabla se encuentra un encabezado de la base de datos que
se trabajara, esta consta de 30 covariables predictoras, las cuales
estarán desde NIR1 hasta NIR30. De primera mano se observa que los
valores de los NIR disminuyen a medida que la covariable aumenta

\begin{Shaded}
\begin{Highlighting}[]
\FunctionTok{library}\NormalTok{(readxl)}
\NormalTok{data }\OtherTok{\textless{}{-}} \FunctionTok{read\_excel}\NormalTok{(}\StringTok{"C:/Users/david/OneDrive/Escritorio/Octavo semestre github/OctavoSemestre/Estadística Aplicada II/Base de datos/data.xlsx"}\NormalTok{)}
\NormalTok{X }\OtherTok{\textless{}{-}}\NormalTok{ data}
\NormalTok{X}\OtherTok{\textless{}{-}} \FunctionTok{cbind}\NormalTok{(X[,}\DecValTok{1}\SpecialCharTok{:}\DecValTok{30}\NormalTok{],X[,}\FunctionTok{colnames}\NormalTok{(X)}\SpecialCharTok{==}\StringTok{\textquotesingle{}density\textquotesingle{}}\NormalTok{])}
\FunctionTok{xtable}\NormalTok{(}\FunctionTok{head}\NormalTok{(X[,}\DecValTok{1}\SpecialCharTok{:}\DecValTok{11}\NormalTok{]))}
\end{Highlighting}
\end{Shaded}

\% latex table generated in R 4.3.0 by xtable 1.8-4 package \% Thu Apr
27 11:56:22 2023

\begin{table}[ht]
\centering
\begin{tabular}{rrrrrrrrrrrr}
  \hline
 & NIR1 & NIR2 & NIR3 & NIR4 & NIR5 & NIR6 & NIR7 & NIR8 & NIR9 & NIR10 & NIR11 \\ 
  \hline
1 & 3.07 & 3.09 & 3.11 & 3.10 & 3.00 & 2.83 & 2.62 & 2.40 & 2.19 & 2.01 & 1.84 \\ 
  2 & 3.07 & 3.09 & 3.10 & 3.07 & 2.98 & 2.84 & 2.68 & 2.51 & 2.35 & 2.22 & 2.12 \\ 
  3 & 3.08 & 3.10 & 3.09 & 3.03 & 2.88 & 2.69 & 2.48 & 2.27 & 2.08 & 1.92 & 1.77 \\ 
  4 & 3.08 & 3.10 & 3.10 & 3.07 & 2.99 & 2.87 & 2.74 & 2.61 & 2.50 & 2.42 & 2.38 \\ 
  5 & 3.10 & 3.10 & 3.08 & 3.02 & 2.89 & 2.72 & 2.54 & 2.38 & 2.24 & 2.13 & 2.05 \\ 
  6 & 3.08 & 3.08 & 3.05 & 2.93 & 2.73 & 2.51 & 2.29 & 2.10 & 1.93 & 1.79 & 1.67 \\ 
   \hline
\end{tabular}
\end{table}

\begin{Shaded}
\begin{Highlighting}[]
\FunctionTok{xtable}\NormalTok{(}\FunctionTok{head}\NormalTok{(X[,}\DecValTok{12}\SpecialCharTok{:}\DecValTok{21}\NormalTok{]))}
\end{Highlighting}
\end{Shaded}

\% latex table generated in R 4.3.0 by xtable 1.8-4 package \% Thu Apr
27 11:56:22 2023

\begin{table}[ht]
\centering
\begin{tabular}{rrrrrrrrrrr}
  \hline
 & NIR12 & NIR13 & NIR14 & NIR15 & NIR16 & NIR17 & NIR18 & NIR19 & NIR20 & NIR21 \\ 
  \hline
1 & 1.69 & 1.58 & 1.50 & 1.44 & 1.34 & 1.22 & 1.14 & 1.12 & 1.13 & 1.16 \\ 
  2 & 2.04 & 1.98 & 1.96 & 1.94 & 1.89 & 1.82 & 1.75 & 1.71 & 1.68 & 1.65 \\ 
  3 & 1.65 & 1.55 & 1.49 & 1.44 & 1.35 & 1.26 & 1.20 & 1.18 & 1.19 & 1.21 \\ 
  4 & 2.35 & 2.35 & 2.37 & 2.40 & 2.40 & 2.38 & 2.33 & 2.28 & 2.21 & 2.11 \\ 
  5 & 1.99 & 1.95 & 1.94 & 1.93 & 1.90 & 1.85 & 1.80 & 1.76 & 1.73 & 1.68 \\ 
  6 & 1.56 & 1.48 & 1.43 & 1.39 & 1.32 & 1.25 & 1.20 & 1.19 & 1.19 & 1.19 \\ 
   \hline
\end{tabular}
\end{table}

\begin{Shaded}
\begin{Highlighting}[]
\FunctionTok{xtable}\NormalTok{(}\FunctionTok{head}\NormalTok{(X[,}\DecValTok{22}\SpecialCharTok{:}\DecValTok{31}\NormalTok{]))}
\end{Highlighting}
\end{Shaded}

\% latex table generated in R 4.3.0 by xtable 1.8-4 package \% Thu Apr
27 11:56:22 2023

\begin{table}[ht]
\centering
\begin{tabular}{rrrrrrrrrrr}
  \hline
 & NIR22 & NIR23 & NIR24 & NIR25 & NIR26 & NIR27 & NIR28 & NIR29 & NIR30 & density \\ 
  \hline
1 & 1.16 & 1.15 & 1.15 & 1.13 & 1.07 & 1.02 & 1.01 & 1.03 & 1.08 & 100.00 \\ 
  2 & 1.58 & 1.51 & 1.45 & 1.38 & 1.29 & 1.20 & 1.15 & 1.13 & 1.14 & 80.22 \\ 
  3 & 1.20 & 1.18 & 1.17 & 1.15 & 1.10 & 1.07 & 1.06 & 1.08 & 1.12 & 79.49 \\ 
  4 & 1.98 & 1.85 & 1.75 & 1.63 & 1.51 & 1.40 & 1.30 & 1.23 & 1.20 & 60.80 \\ 
  5 & 1.60 & 1.52 & 1.46 & 1.39 & 1.31 & 1.24 & 1.19 & 1.16 & 1.17 & 59.97 \\ 
  6 & 1.18 & 1.15 & 1.14 & 1.12 & 1.09 & 1.06 & 1.06 & 1.07 & 1.11 & 60.48 \\ 
   \hline
\end{tabular}
\end{table}

\hypertarget{selecciuxf3n-de-variables}{%
\subsection{Selección de variables}\label{selecciuxf3n-de-variables}}

En el proceso de selección de variables se procede a ajustar todos los
posibles modelos \(2^{27}\), del cual se observa el \(R^2_{adj}\), el
AIC y el BIC. Además se realizan los algoritmos de selección (forward
selection, backward selection, stepwise selection) y Regresion de LASSO.
Luego se ajustara el modelo con las variables que tengan buenos
indicadores y ademas que permita corregir supuestos.

\hypertarget{todos-los-posibles-modelos}{%
\section{Todos los posibles modelos}\label{todos-los-posibles-modelos}}

Con la funcion ols\_step\_all\_possible() de la librería olsrr es
posible ajustar todos los posibles modelos y determinar el mejor bajo
diferentes criterios. Para ajustar todos los posibles modelos es
necesario tener los suficientes grados de libertad para calcular las
estimaciones. Por ende no se tomaran las 30 variables

\begin{Shaded}
\begin{Highlighting}[]
\FunctionTok{library}\NormalTok{(olsrr)}
\NormalTok{model }\OtherTok{\textless{}{-}} \FunctionTok{lm}\NormalTok{(density }\SpecialCharTok{\textasciitilde{}}\NormalTok{., }\AttributeTok{data=}\NormalTok{X)}
\FunctionTok{summary}\NormalTok{(model)}
\end{Highlighting}
\end{Shaded}

Call: lm(formula = density \textasciitilde{} ., data = X)

Residuals: ALL 28 residuals are 0: no residual degrees of freedom!

Coefficients: (3 not defined because of singularities) Estimate Std.
Error t value Pr(\textgreater\textbar t\textbar) (Intercept) -859.7 NaN
NaN NaN NIR1 -634.7 NaN NaN NaN NIR2 1823.8 NaN NaN NaN NIR3 -1934.0 NaN
NaN NaN NIR4 2943.4 NaN NaN NaN NIR5 -6923.8 NaN NaN NaN NIR6 15592.6
NaN NaN NaN NIR7 -19517.0 NaN NaN NaN NIR8 7601.3 NaN NaN NaN NIR9
-1279.5 NaN NaN NaN NIR10 1685.3 NaN NaN NaN NIR11 5057.9 NaN NaN NaN
NIR12 -7777.7 NaN NaN NaN NIR13 22522.6 NaN NaN NaN NIR14 -12342.9 NaN
NaN NaN NIR15 -4469.0 NaN NaN NaN NIR16 -2338.2 NaN NaN NaN NIR17 4153.6
NaN NaN NaN NIR18 -17982.1 NaN NaN NaN NIR19 35565.4 NaN NaN NaN NIR20
-22850.4 NaN NaN NaN NIR21 5711.8 NaN NaN NaN NIR22 -2781.1 NaN NaN NaN
NIR23 -22323.0 NaN NaN NaN NIR24 54050.3 NaN NaN NaN NIR25 -50489.5 NaN
NaN NaN NIR26 23668.7 NaN NaN NaN NIR27 -6842.9 NaN NaN NaN NIR28 NA NA
NA NA NIR29 NA NA NA NA NIR30 NA NA NA NA

Residual standard error: NaN on 0 degrees of freedom Multiple R-squared:
1, Adjusted R-squared: NaN F-statistic: NaN on 27 and 0 DF, p-value: NA

\end{document}
